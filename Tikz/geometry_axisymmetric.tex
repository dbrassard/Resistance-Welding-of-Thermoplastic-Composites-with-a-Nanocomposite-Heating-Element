%\shorthandoff{:!}

\begin{tikzpicture}[scale=0.038]

%%%%%%%%%%%%%%%%%%%%%%
%%%	 Def géométrie  %%%
%%%%%%%%%%%%%%%%%%%%%%

\def \rayonun{6}
\def \rayondeux{65}
\def \vspace{2}
\def \lignecote{8}
\def \hauteur{80}
\def \anglearc{111.5}

%%%%%%%%%%%%%%%%%%%%%%
%%%	   Def style    %%%
%%%%%%%%%%%%%%%%%%%%%%

\def \heavy{0.045cm}
\def \light{0.025cm}

%%%%%%%%%%%%%%%%%%%%%%
%%%	    Calculs     %%%
%%%%%%%%%%%%%%%%%%%%%%

\pgfmathsetmacro{\sinangle}{sin(\anglearc)}
\pgfmathsetmacro{\cosangle}{cos(\anglearc)}

%%%%%%%%%%%%%%%%%%%%%%
%%%	    Dessin      %%%
%%%%%%%%%%%%%%%%%%%%%%

\draw[line width=\heavy] (\rayonun,0) rectangle (\rayondeux,\hauteur);
\draw[line width=\heavy, fill=gray] (0,0) rectangle (\rayonun,\hauteur);
%\draw[line width=\heavy] \draw (0,0,0) arc (0:45:1) ;

% On se positionne sur un plan à une hauteur de 80 (\hauteur)
\begin{scope}[canvas is zx plane at y=\hauteur]
	% On dessine le quart de cercle extérieur
	\draw[line width=\heavy] (0,\rayondeux) arc (90:180:\rayondeux) -- (0,0);
	% On dessine le quart de cercle du nanotube
	\draw[line width=\heavy, fill=gray] (0,0) -- (0,\rayonun) arc (90:180:\rayonun) -- (0,0);
\end{scope}

% On se positionne sur un plan à une hauteur de 0
\begin{scope}[canvas is zx plane at y=0]
	% On trace le cercle jusqu'à l'angle \anglearc trouvé itérativement
	\draw[line width=\heavy] (0,\rayondeux) arc (90:\anglearc:\rayondeux);
\end{scope}

% Ligne du côté du cylindre
\draw[line width=\heavy] (\sinangle*\rayondeux,0,\cosangle*\rayondeux) -- +(0,\hauteur,0);

%%%%%%%%%%%%%%%%%%%%%%
%%%	    Cotation    %%%
%%%%%%%%%%%%%%%%%%%%%%

% On dessine la ligne de cote pour la nanotube
\draw[line width=\light] (0,-\vspace) -- + (0,-\lignecote);
\draw[line width=\light] (\rayonun,-\vspace) -- + (0,-\lignecote);
\draw[<->,line width=\heavy] (0,-\vspace-0.5*\lignecote) -- + (\rayonun,0);

\node[anchor=north] (A) at (0.5*\rayonun , -\vspace-\lignecote) {\footnotesize \rayonun \ nm};

% On dessine la ligne de cote pour la hauteur
\draw[line width=\light] (-\vspace,0) -- + (-\lignecote,0);
\draw[line width=\light] (-\vspace,\hauteur) -- + (-\lignecote,0);
\draw[<->,line width=\heavy] (-\vspace-0.5*\lignecote,0) -- + (0,\hauteur);

\node[anchor=south, rotate=90] (A) at (-\vspace-\lignecote,0.5*\hauteur ) {\footnotesize \hauteur \ nm};

% On dessine la ligne de cote pour le diamètre du modèle
\draw[line width=\light] (0,\hauteur+\vspace) -- + (0,\lignecote);
\draw[line width=\light] (\rayondeux,\hauteur+\vspace) -- + (0,\lignecote);
\draw[<->,line width=\heavy] (0,\hauteur+\vspace+0.5*\lignecote) -- + (\rayondeux,0);

\node[anchor=south] (A) at (0.5*\rayondeux , \hauteur+\vspace+\lignecote) {\footnotesize \rayondeux \ nm};

%%%%%%%%%%%%%%%%%%%%%%%%%%%%%%%%%%%%%%%%%

\end{tikzpicture}

%\shorthandon{:!}